\documentclass[12pt]{article}

\usepackage{graphicx}
\usepackage{paralist}
\usepackage{amsfonts}
\usepackage{amsmath}
\usepackage{hhline}
\usepackage{booktabs}
\usepackage{multirow}
\usepackage{multicol}
\usepackage{url}
\usepackage{hyperref}

\oddsidemargin -10mm
\evensidemargin -10mm
\textwidth 160mm
\textheight 200mm
\renewcommand\baselinestretch{1.0}

\pagestyle {plain}
\pagenumbering{arabic}

\newcounter{stepnum}

%% Comments

\usepackage{color}

\newif\ifcomments\commentstrue

\ifcomments
\newcommand{\authornote}[3]{\textcolor{#1}{[#3 ---#2]}}
\newcommand{\todo}[1]{\textcolor{red}{[TODO: #1]}}
\else
\newcommand{\authornote}[3]{}
\newcommand{\todo}[1]{}
\fi

\newcommand{\wss}[1]{\authornote{blue}{SS}{#1}}

\title{Assignment 3, Part 1, Specification}
\author{SFWR ENG 2AA4, COMP SCI 2ME3}

\begin {document}

\maketitle
This Module Interface Specification (MIS) document contains modules, types and
methods used to support the CEAB (Canadian Engineering Accreditation Board)
graduate attributes process.  An accredited program needs to measure the
learning outcomes in all of the courses and show that the measures adequately
satisfy the graduate attributes specified by the CEAB.

For every learning outcome in every course in a given program the number of
students is counted in each of the following categories: below expectations,
marginal, meets expectations and exceeds expectations.  This data for each
learning outcome needs to be aggregated to express the results using the CEAB's
graduate attributes.  We will use the same categories when summarizing the
attributes, but rather than using student numbers, below, marginal, meets and
exceeds will be expressed as percentages.

Each attribute is divided into a set of indicators.  Each program consists of a
set of courses and each course will cover a set of indicators.  For each
indicator in a course there will be a set of learning outcomes.  The learning
outcomes are course specific, but the indicators and the attributes are
determined by the CEAB and the Faculty of Engineering, respectively.

There are many choices on how to aggregate the data, depending on at what point
the data is normalized.  These modules allow this choice to be modified at
run-time so that the different possibilities can be dynamically explored.  This
is enabled by the use of the Norm abstract object module.  At one extreme the
data for a program is only normalized after summing all of the student counts
for each learning outcome for each indicator for each attribute in each course.
At the other extreme all of the data is normalized at each step, starting with
the measures of the learning outcomes themselves.  A comparison of the different
schemes can be found in the
\href{https://gitlab.cas.mcmaster.ca/smiths/se2aa4_cs2me3/-/blob/master/Assignments/A3/AggregationStrategies.xlsx}
{sample spreadsheet} available in the assignment folder.

% In applying the specification, there may be cases that involve undefinedness.
% We will interpret undefinedness following~\cite{Farmer2004}:

% If $p: \alpha_1 \times .... \times \alpha_n \rightarrow \mathbb{B}$ and any of
% $a_1, ..., a_n$ is undefined, then $p(a_1, ..., a_n)$ is False.  For instance,
% if $p(x) = 1/x < 1$, then $p(0) = \text{False}$.  In the language of our
% specification, if evaluating an expression generates an exception, then the
% value of the expression is undefined.

% \wss{The parts that you need to fill in are marked by comments, like this one.
%   In several of the modules local functions are specified.  You can use these
%   local functions to complete the missing specifications.}

% \wss{As you edit the tex source, please leave the \texttt{wss} comments in the
%   file.  Put your answer \textbf{after} the comment.  This will make grading
%   easier.}

% \bibliographystyle{plain}
% \bibliography{SmithCollectedRefs}

\newpage

\section* {IndicatorT Module}

\subsection*{Module}

IndicatorT

\subsection* {Uses}

None

\subsection* {Syntax}

\subsubsection* {Exported Constants}

None

\subsubsection* {Exported Types}

IndicatorT = \{\\
    math, \textit{\#Competence in mathematics}\\
    specEngKnow, \textit{\#Competence in specialized engineering knowledge}\\
    assumpt, \textit{\#Ability to identify reasonable assumptions}\\
    suitableFund, \textit{\#Ability to identify suitable engineering fundamentals}\\
    recogTheory, \textit{\#Able to recognize and discuss applicable theory knowledge base}\\
    modelSelect, \textit{\#Selects appropriate model and methods}\\
    estOutcomes, \textit{\#Estimates outcomes, uncertainties and determines data to collect}\\
    desProcess, \textit{\#Recognizes and follows an engineering design process}\\
    desPrinciples, \textit{\#Recognizes and follows engineering design principles}\\
    openEnded, \textit{\#Proposes solutions to open-ended problems}\\
    ideaGeneration, \textit{\#Employs appropriate techniques for generation of creative ideas}\\
    healthSafety, \textit{\#Includes appropriate health and safety considerations}\\
    standards, \textit{\#Determines and employs applicable standards and codes of practice}\\
    tools, \textit{\#The ability to use modern/state of the art tools}\\
    engInSoc, \textit{\#Demonstrates an understanding of the role of the engineer in society}\\
    awarePEO; \textit{\#Aware of PEO and role of licensing}\\
\}

\subsubsection* {Exported Access Programs}

None

\subsection* {Semantics}

\subsubsection* {State Variables}

None

\subsubsection* {State Invariant}

None

\subsubsection* {Assumptions}

None

\subsubsection* {Considerations}

When implementing in Java, use enums (as shown in Tutorial 07 for ElementT).

\newpage

\section* {Attributes Module}

\subsection*{Template Module}

AttributeT

\subsection* {Uses}

IndicatorT

\subsection* {Syntax}

\subsubsection* {Exported Constants}

None

\subsubsection* {Exported Types}

AttributeT = ?

\subsubsection* {Exported Access Programs}

\begin{tabular}{| l | l | l | p{5cm} |}
\hline
\textbf{Routine name} & \textbf{In} & \textbf{Out} & \textbf{Exceptions}\\
\hline
new AttributeT & String, seq of IndicatorT & AttributeT & \\
\hline
getName & & String & ~\\
\hline
getIndicators & & seq of IndicatorT & ~\\
\hline
\end{tabular}

\subsection* {Semantics}

\subsubsection* {State Variables}

$\mathit{name}: \text{String}$\\
$s: \text{set of IndicatorT}$

\subsubsection* {State Invariant}

None

\subsubsection* {Assumptions}

None

\subsubsection* {Access Routine Semantics}

\noindent new AttributeT($\mathit{attribName}, \mathit{indicators}$):
\begin{itemize}
\item transition: $\mathit{name}, \mathit{s} := \mathit{attribName}, \cup ( i:
  \text{IndicatorT} | i \in \mathit{indicators} : \{ i \} )$
\item output: $out := \mbox{self}$
\item exception: none
\end{itemize}

\noindent getName():
\begin{itemize}
\item output: $out := \mathit{name}$
\item exception: none
\end{itemize}

\noindent getIndicators():
\begin{itemize}
\item output: $out := \langle i: \text{IndicatorT} | i \in s : i \rangle$
  \textit{\# order in the sequence does not matter}
\item exception: none
\end{itemize}

\newpage

\section* {Measures Interface Module}

\subsection*{Interface Module}

Measures

\subsection* {Uses}

IndicatorT, AttributeT

\subsection* {Syntax}

\subsubsection* {Exported Constants}

None

\subsubsection* {Exported Types}

None 

\subsubsection* {Exported Access Programs}

\begin{tabular}{| l | l | l | p{6cm} |}
\hline
\textbf{Routine name} & \textbf{In} & \textbf{Out} & \textbf{Exceptions}\\
\hline
measures & & seq of $\mathbb{R}$ & UnsupportedOperationException\\
\hline
measures & ind: IndicatorT & seq of $\mathbb{R}$ & UnsupportedOperationException\\
\hline
measures & att: AttributeT & seq of $\mathbb{R}$ & UnsupportedOperationException\\
\hline
  
\end{tabular}
    
\subsubsection* {Considerations}

The interface provides several signatures for the measures method.  The modules
that inherit Measures should provide all versions.  The
UnsupportedOperationException is included because not all signatures are valid
for every subclass.

The entries in the output sequence at indices 0 to 3 correspond respectively to
the number of measures: below expectations, marginal, meets expectations,
exceeds expectations.

\newpage

\section* {Services Module}

\subsection*{Module}

Services

\subsection* {Uses}

None

\subsection* {Syntax}

\subsubsection* {Exported Constants}

None

\subsubsection* {Exported Types}

None

\subsubsection* {Exported Access Programs}

\begin{tabular}{| l | l | l | p{5cm} |}
\hline
\textbf{Routine name} & \textbf{In} & \textbf{Out} & \textbf{Exceptions}\\
\hline
normal & seq of $\mathbb{R}$ & seq of $\mathbb{R}$ & \\
\hline
\end{tabular}

\subsection* {Semantics}

\subsubsection* {State Variables}

None

\subsubsection* {State Invariant}

None

\subsubsection* {Assumptions}

None

\subsubsection* {Access Routine Semantics}

\noindent normal($v$):
\begin{itemize}
\item output: $\mathit{out} := \langle i: \mathbb{N} | i \in [0 .. |v|-1] :
  v_i/\text{sum}(v) \rangle$
\item exception: none
\end{itemize}

\subsection*{Local Functions}

\noindent sum: seq of $\mathbb{R} \rightarrow \mathbb{R}$\\
\noindent $\mbox{sum}(x) \equiv + (i: \mathbb{N} | i \in [0 .. |x|-1] : x_i )$\\

\newpage

\section* {Norm Aggregation Algorithm Configuration Module (Abstract Object)}

\subsection*{Module}

Norm

\subsection* {Uses}

None

\subsection* {Syntax}

\subsubsection* {Exported Constants}

None

\subsubsection* {Exported Types}

None

\subsubsection* {Exported Access Programs}

\begin{tabular}{| l | l | l | p{5cm} |}
  \hline
  \textbf{Routine name} & \textbf{In} & \textbf{Out} & \textbf{Exceptions}\\
  \hline
  setNorms & $\mathbb{B}$, $\mathbb{B}$, $\mathbb{B}$ & & \\
  \hline
  getNLOs & & $\mathbb{B}$ & \\
  \hline
  getNInd & & $\mathbb{B}$ & \\
  \hline
  getNAtt & & $\mathbb{B}$ & \\
  \hline
  setNLOs & $\mathbb{B}$ & & \\
  \hline
  setNInd & $\mathbb{B}$ & & \\
  \hline
  setNAtt & $\mathbb{B}$ & & \\
  \hline
  
\end{tabular}

\subsection* {Semantics}

\subsubsection* {State Variables}

$\mathit{normLOs}$: $\mathbb{B}$\\
$\mathit{normInd}$: $\mathbb{B}$\\
$\mathit{normAtt}$: $\mathbb{B}$

\subsubsection* {State Invariant}

None

\subsubsection* {Assumptions}

The state variables will be set before they are used.

\subsubsection* {Access Routine Semantics}

\noindent setNorms($\mathit{nLOs}$, $\mathit{nInd}$, $\mathit{nAtt}$):
\begin{itemize}
\item transition: $\mathit{normLOs}, \mathit{normInd}, \mathit{normAtt} := \mathit{nLOs}, \mathit{nInd}, \mathit{nAtt}$
\item exception: none
\end{itemize}

\noindent getNLOs( ):
\begin{itemize}
\item output: $out := \mathit{normLOs}$
\item exception: none
\end{itemize}

\noindent getNInd( ):
\begin{itemize}
\item output: $out := \mathit{normInd}$
\item exception: none
\end{itemize}

\noindent getNAtt( ):
\begin{itemize}
\item output: $out := \mathit{normAtt}$
\item exception: none
\end{itemize}

\noindent setNLOs($\mathit{nLOs}$):
\begin{itemize}
\item transition: $\mathit{normLOs} := \mathit{nLOs}$
\item exception: none
\end{itemize}

\noindent setNInd($\mathit{nInd}$):
\begin{itemize}
\item transition: $\mathit{normInd} := \mathit{nInd}$
\item exception: none
\end{itemize}

\noindent setNAtt($\mathit{nAtt}$):
\begin{itemize}
\item transition: $\mathit{normAtt} := \mathit{nAtt}$
\item exception: none
\end{itemize}

\newpage

\section* {Learning Outcomes Module}

\subsection*{Template Module inherits Measures}

LOsT

\subsection* {Uses}

Measures, IndicatorT, AttributeT, Services, Norm

\subsection* {Syntax}

\subsubsection* {Exported Constants}

None

\subsubsection* {Exported Types}

LOsT = ?

\subsubsection* {Exported Access Programs}

\begin{tabular}{| l | l | l | p{5cm} |}
\hline
\textbf{Routine name} & \textbf{In} & \textbf{Out} & \textbf{Exceptions}\\
\hline
new LOsT & String, $\mathbb{Z}$, $\mathbb{Z}$, $\mathbb{Z}$, $\mathbb{Z}$ & LOsT & IllegalArgumentException\\
\hline
getName & & String & ~\\
\hline
equals & LOsT & $\mathbb{B}$ & ~\\
\hline
\end{tabular}

\subsection* {Semantics}

\subsubsection* {State Variables}

$\mathit{name}: \text{String}$\\
$\mathit{n\_blw}$: $\mathbb{N}$\\
$\mathit{n\_mrg}$: $\mathbb{N}$\\
$\mathit{n\_mts}$: $\mathbb{N}$\\
$\mathit{n\_exc}$: $\mathbb{N}$

\subsubsection* {State Invariant}

None

\subsubsection* {Assumptions}

None

\subsubsection* {Access Routine Semantics}

\noindent new LOsT($\mathit{topic}, \mathit{nblw}, \mathit{nmrg}, \mathit{nmts}, \mathit{nexc}$):
\begin{itemize}
\item transition: $\mathit{name}, \mathit{n\_blw}, \mathit{n\_mrg},
  \mathit{n\_mts}, \mathit{n\_exc} := \mathit{topic}, \mathit{nblw},
  \mathit{nmrg}, \mathit{nmts}, \mathit{nexc}$
\item output: $out := \mathit{self}$
\item exception: exc := $((\mathit{nblw} < 0) \vee (\mathit{nmrg} < 0) \vee 
  (\mathit{nmts} < 0) \vee (\mathit{nexc}<0) \Rightarrow
  \text{IllegalArgumentException}| (\mathit{nblw} = 0) \wedge (\mathit{nmrg} = 0) \wedge 
  (\mathit{nmts} = 0) \wedge (\mathit{nexc}=0) \Rightarrow \text{IllegalArgumentException})$
\end{itemize}

\noindent getName():
\begin{itemize}
\item output: $out := \mathit{name}$
\item exception: none
\end{itemize}

\noindent equals($o$):
\begin{itemize}
\item output: $out := \mathit{name} = o.\text{getName}()$
\item exception: none
\end{itemize}

\noindent measures($\mathit{}$):
\begin{itemize} 
\item output: \textit{\# output should also be converted from natural numbers to reals}
  $$out := (\neg \text{Norm.getNLOs()} \Rightarrow [\mathit{n\_blw}, \mathit{n\_mrg}, \mathit{n\_mts},
  \mathit{n\_exc}] | \text{True} \Rightarrow \text{normal}([\mathit{n\_blw},
  \mathit{n\_mrg}, \mathit{n\_mts}, \mathit{n\_exc}]))$$
  
\item exception: none
\end{itemize}

\noindent measures($\mathit{ind}$: IndicatorT):
\begin{itemize}
\item output: none
\item exception: exc := UnsupportedOperationException
\end{itemize}

\noindent measures($\mathit{att}$: AttributeT):
\begin{itemize}
\item output: none
\item exception: exc := UnsupportedOperationException
\end{itemize}

\newpage

\section* {Courses Module}

\subsection*{Template Module inherits Measures}

CourseT

\subsection* {Uses}

Measures, IndicatorT, AttributeT, LOsT, Services, Norm

\subsection* {Syntax}

\subsubsection* {Exported Constants}

None

\subsubsection* {Exported Types}

CourseT = ?

\subsubsection* {Exported Access Programs}

\begin{tabular}{| l | l | l | p{5cm} |}
  \hline
  \textbf{Routine name} & \textbf{In} & \textbf{Out} & \textbf{Exceptions}\\
  \hline
  new CourseT & String, seq of IndicatorT & CourseT & \\
  \hline
  getName & & String & ~\\
  \hline
  getIndicators & & seq of IndicatorT & ~\\
  \hline
  getLOs & IndicatorT & seq of LOsT & ~\\
  \hline
  addLO & IndicatorT, LOsT & & ~\\
  \hline
  delLO & IndicatorT, LOsT & & ~\\
  \hline
  member & IndicatorT, seq of LOsT & $\mathbb{B}$ & ~\\
  \hline
  
\end{tabular}

\subsection* {Semantics}

\subsubsection* {State Variables}

$\mathit{name}: \text{String}$\\
$m$: set of MapInd2LOsT \textit{\# mapping between indicators and set of learning outcomes}
%$\mathit{measDone}$: $\mathbb{B}$

\subsubsection* {State Invariant}

None

\subsubsection* {Assumptions}

The programmer will not use CourseT until the indicators have been populate with
learning outcomes.  The constructor starts with the set of learning outcomes
being empty; therefore, addLO needs to be called at least once for each
indicator for the course.  (\textit{\# Initially a measureDone flag was part of
  the state variables for the module to track this, but in the interest of
  keeping things simpler, it was removed.})  If learning outcomes are removed,
it is assumed that not all of them will be removed before the Course is used in
an aggregation calculation.

\subsubsection* {Access Routine Semantics}

\noindent new CourseT($\mathit{courseName}$, $\mathit{indicators}$):
\begin{itemize}
\item transition: $\mathit{name}, m := \mathit{courseName}, \{i: \text{IndicatorT} |
  i \in \mathit{indicators} : \langle i, \{ \} \rangle \}$
\item output: $out := \mbox{self}$
\item exception: none
\end{itemize}

\noindent getName():
\begin{itemize}
\item output: $out := \mathit{name}$
\item exception: none
\end{itemize}

\noindent getIndicators():
\begin{itemize}
\item output: $out := \langle s: \text{MapInd2LOsT } | s \in m : s.\mathit{ind} \rangle$
\item exception: none
\end{itemize}

\noindent getLOs($\mathit{indicator}$):
\begin{itemize}
\item output:
  $out := (\langle \mathit{indicator}, \mathit{LOs} \rangle \not\in m
  \Rightarrow [] | \text{True} \Rightarrow \text{set\_to\_seq}(\mathit{LOs})
  \text{ where } \langle \mathit{indicator}, \mathit{LOs} \rangle \in m)$
\item exception: none
\end{itemize}

\noindent addLO($\mathit{indicator}$, $\mathit{outcome}$):
\begin{itemize}
\item transition:
  $$m := \{ s: \text{MapInd2LOsT} | s \in m : (s.\mathit{ind} =
    \mathit{indicator} \Rightarrow \langle s.\mathit{ind}, s.\mathit{LOs} \cup
    \{\mathit{outcome}\} \rangle | \text{True} \Rightarrow s) \}$$
\item exception: none
\end{itemize}

\noindent delLO($\mathit{indicator}$, $\mathit{outcome}$):
\begin{itemize}
\item transition:
  $$m := \{ s: \text{MapInd2LOsT} | s \in m : (s.\mathit{ind} =
    \mathit{indicator} \Rightarrow \langle s.\mathit{ind}, s.\mathit{LOs} -
    \{\mathit{outcome}\} \rangle | \text{True} \Rightarrow s) \}$$
\item exception: none
\end{itemize}

\noindent member($\mathit{indicator}$, $\mathit{outcomes}$):
\begin{itemize}
\item output:
  $$out := \exists(s: \text{MapInd2LOsT} | s \in m : s.\mathit{ind} =
  \mathit{indicator} \wedge \forall ( x: \text{LOsT} | : x \in s.\mathit{LOs}
  \leftrightarrow x \in \mathit{outcomes}))$$
\item exception: none
\end{itemize}

\noindent measures():
\begin{itemize}
\item output: none
\item exception: exc := UnsupportedOperationException
\end{itemize}

\noindent measures($\mathit{ind}$: IndicatorT):
\begin{itemize}
\item output:
  $out := (\mathit{self}.\text{getLOs}(\mathit{ind}) = [] \Rightarrow [0, 0,  0,
  0] | \text{True} \Rightarrow (\text{Norm.getNInd()} \Rightarrow \text{normal}(\mathit{measureInd})
  | \text{True} \Rightarrow \mathit{measureInd} ))$ \newline where $\mathit{measureInd}
  \equiv \text{sumMeas}(\mathit{o}: \text{LOsT} | \mathit{o} \in
  \mathit{self}.\text{getLOs}(\mathit{ind}) :
  \mathit{o}.\text{measures}())$

  \textit{\# sumMeas is a binary operator, so it can be used with our normal
    notation.}
  
\item exception: none
\end{itemize}

\noindent measures($\mathit{att}$: AttributeT):
\begin{itemize}
\item output:
  $out := (\mathit{att}.\text{getIndicators()} = [] \Rightarrow [0, 0, 0, 0] |
  \text{True} \Rightarrow (\text{Norm.getNAtt()} \Rightarrow \text{normal} (\mathit{measureInd})
  | \text{True} \Rightarrow \mathit{measureInd} ))$ \newline where $\mathit{measureInd}
  \equiv \text{sumMeas}(\mathit{i}: \text{IndicatorT} | \mathit{i} \in
  \mathit{att}.\text{getIndicators()} :
  \mathit{self}.\text{measures}(i))$
\item exception: none
\end{itemize}

\subsection*{Local Functions}

\noindent $\mbox{set\_to\_seq}: \text{set of } \text{LOsT} \rightarrow \mbox{seq of }
\text{LOsT}$\\
\noindent
$\mbox{set\_to\_seq}(s) \equiv \langle x: \text{LOsT} | x \in s : x \rangle$
\textit{\# Return a seq of all of the elems in the set s}
~\\

\noindent $\mbox{sumMeas}: \text{seq [4] of } \mathbb{R} \times \text{seq
  [4] of } \mathbb{R} \rightarrow \text{seq [4] of } \mathbb{R}$\\
\noindent
$\mbox{sumMeas}(a, b) \equiv [a_0 + b_0, a_1 + b_1, a_2 + b_2, a_3 + b_3]$

\subsection*{Local Types}

\noindent MapInd2LOsT = tuple of ($\mathit{ind}$: IndicatorT, $\mathit{LOs}$: set
of LOsT)

\newpage

\section* {ProgramT Module}

ProgramT

\subsection* {Template Module inherits Measure, HashSet(CourseT)}

\subsection* {Uses}

Measures, IndicatorT, AttributeT, Services, Norm, CourseT, HashSet

\subsection* {Syntax}

\subsubsection* {Exported Constants}

None

\subsubsection* {Exported Types}

ProgramT = ?

\subsubsection* {Exported Access Programs}

\begin{tabular}{| l | l | l | p{5cm} |}
  \hline
  \textbf{Routine name} & \textbf{In} & \textbf{Out} & \textbf{Exceptions}\\
  \hline
  \hline
  
\end{tabular}

\subsection* {Semantics}

\subsubsection* {State Variables}

s: set of CourseT \textit{\# inherited from HashSet(CourseT)}

\subsubsection* {State Invariant}

\subsubsection* {Assumptions}

The set of courses will not be empty and at least one course will have non-zero
measures for each attribute.

\subsubsection* {Access Routine Semantics}

\noindent measures():
\begin{itemize}
\item output: none
\item exception: exc := UnsupportedOperationException
\end{itemize}

\noindent measures($\mathit{ind}$: IndicatorT):
\begin{itemize}
\item output: none
\item exception: exc := UnsupportedOperationException
\end{itemize}

\noindent measures($\mathit{att}$: AttributeT):
\begin{itemize}
\item output:
  $out := \text{normal} (\text{sumMeas}(c: \text{CourseT} | c \in s :
  \mathit{c}.\text{measures}(att)))$
\item exception: none
\end{itemize}

\newpage

\section* {HashSet Module}

\subsection* {Generic Template Module inherits Set(E)}

HashSet(E)

\subsection* {Considerations}

Implemented as part of Java, as described in the
\href{https://docs.oracle.com/javase/8/docs/api/java/util/HashSet.html} {Oracle Documentation}

\end {document}